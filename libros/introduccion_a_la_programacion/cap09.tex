\chapter{9na semana - Repaso y Ejercicios}
\section{Guía de repaso}
\nota{Objetivo}{El objetivo de esta guía es ayudarte a buscar cómo resolver pedazos de ejercicios, trabajos prácticos o bien en cualquier momento que te quieras hacer un programa en Ruby, por eso el equipo de Nahual te armó este documento}
\nota{Como usar la guía}{La guía está escrita de una manera amigable para que te sea fácil buscar lo que querés, la mayoría de los títulos van a ser sentencias, y vamos a tratar de que tengan también ejemplos para que puedas usarlas}

La guía arranca de lo más fácil a lo más complejo, por lo tanto, si vos estás buscando algo como iteradores o condiciones que son las últimas cosas que vimos, es probable que las encuentres cerca del final, y si buscás cómo convertir datos o cómo transformar un numero en una letra y cosas así que vimos al principio, van a estar en la primer parte.

\subsection{Mostrar un mensaje en la pantalla}
Cuando un ejercicio nos dice, por ejemplo, que tenemos que poder elevar un numero a cualquier potencia, lo más básico es pensar que vamos a mostrarle al usuario un mensaje que diga, “ingrese un numero:”, esto se hace en Ruby con:
\begin{lstlisting}
puts 'Ingrese un numero'
\end{lstlisting}

\subsection{Mostrar un número en pantalla}
\begin{lstlisting}
puts 5
\end{lstlisting}

\subsection{Recibir un dato de la pantalla}
Cuando quiera guardar un dato que me ingresan, lo hago en una variable, por ejemplo numero, entonces tengo que hacer:
\begin{lstlisting}
puts 'Ingrese un numero'
numero = gets.chomp
\end{lstlisting}

Y ahí el valor de lo que el usuario escriba por pantalla queda en la variable numero, por ejemplo

\begin{lstlisting}
puts 'Ingrese su nombre'
nombre = gets.chomp
\end{lstlisting}
\subsection{Convertir un número a letra}
Muchas veces tenemos que escribir un mensaje, por ejemplo, que sea "el resultado es 2" y esto es bastante complicado porque 2 es un número, entonces hacemos lo siguiente

\begin{lstlisting}
2.to_s
\end{lstlisting}

\subsection{Convertir una letra a número}
Cuando recibimos los datos desde la pantalla son solo letras, entonces de golpe uno puede querer convertirlos a número, entonces hace:

\begin{lstlisting}
variable.to_i
\end{lstlisting}

\subsection{Hacer algo según una condición}
Si se cumple cierta condición tenemos que hacer algo, sino otra cosa:

\begin{lstlisting}
if (condicion)
  hacemos lo que se hace cuando la condicion es verdadera 
else
  hacemos lo que se hace cuando es falsa
end
\end{lstlisting}

\subsection{Repetir algo mientras se cumple una condición}
A veces queremos poder repetir algo mientras se cumple una cierta condición

\begin{lstlisting}
while{condicion}
 hacer lo que queremos hacer muchas veces
end
\end{lstlisting}

\section{Cuestionario chiquito}
\nota{Objetivo}{Necesitamos ir aprendiendo constantemente y para eso a veces es bueno poder repasar un poquito las cosas que aprendemos, como hoy es un día de repaso vamos a hacer un cuestionario cortito que nos sirve para saber cómo venimos, aclaramos que EL CUESTIONARIO NO LLEVA NOMBRE Y NO ES UNA PRUEBA solamente es para saber dónde estamos parados}

\subsection{Números y Letras}
\begin{enumerate}
  \item ¿Qué resultado devuelve hacer 2 + 3?
  \item ¿Qué resultado devuelve hacer 2.0 + 3?
  \item ¿Qué resultado devuelve hacer 2 + 3.5?
  \item ¿Qué resultado devuelve hacer 'Hola' + ' Mariano'?
\end{enumerate}

\subsection{Condiciones}
\begin{enumerate}
  \item ¿Qué resultado devuelve 3 == 3?
  \item ¿Qué resultado devuelve 3 > 3?
  \item ¿Qué resultado devuelve 3 >= 3?
  \item ¿Qué resultado devuelve 'Hola' >= 'Mariano'?
  \item ¿Qué resultado devuelve 'Hola' >= 'mariano'?
\end{enumerate}

\subsection{Funciones especiales}
\begin{enumerate}
  \item ¿Cómo transformamos de número a letra?
  \item ¿Cómo transformamos de mayúscula a minúscula?
  \item ¿Cómo transformamos de letra a número?
\end{enumerate}

\subsection{Variables}
\begin{enumerate}
  \item ¿Qué estamos haciendo cuando hacemos variable = 'Mariano'?
  \item ¿Qué devuelve el siguiente programita?
\end{enumerate}

\begin{lstlisting}
mi_nombre = 'Charly'
mi_apellido = 'Lizarralde'
puts 'Yo me llamo ' + mi_nombre + '  y mi apellido es ' + mi_apellido
\end{lstlisting}

\subsection{Loops}
En el siguiente programa:

\begin{lstlisting}
contador = 20
while contador < 30
  puts contador
  contador = contador + 2
end
\end{lstlisting}
\begin{enumerate}
  \item ¿Con qué valor se inicializa el contador?
  \item ¿Cuál es la condición?
  \item ¿En cuanto se incrementa el contador?
  \item ¿Qué devuelve el programa?
\end{enumerate}

\section{Ejercicios de repaso}
\nota{Objetivo}{El objetivo de esta guía es repasar algunos ejercicios que fuimos haciendo y juntarlos, sobre todo los últimos de condiciones y ciclos para que podamos seguir practicando ¡ya que ahora tenemos días para hacer ejercicios!}
\subsection{Loops}
\nota{Pistas}{Tenemos que definir las partes de un ciclo repetitivo o loop ¿se acuerdan? la condición, el valor inicial y como incrementamos la variable que vamos sumando}
\begin{enumerate}
  \item Hacer un programa que muestre los números que van del 30 al 50
  \item Hacer un programa que muestre los números pares entre 1 y 70
  \item Hacer un programa que muestre los números impares entre 2 y 90
  \item Hacer un programa que muestre los números entre 1 y 70, pero al revés, o sea empezando por el 70, siguiendo por el 69, 68.
\end{enumerate}

\subsection{Loops, pero con condiciones más difíciles}
\nota{Pistas}{Ahora vamos a tener que pensar un poco más cuándo terminamos de iterar y cómo calculamos}

Resulta que El Pichuleo, un joven atorrante de Entre Ríos, se fue a comprar ropa a la feria de la salada, pero solo llevó 100 pesos. Necesita comprar varias prendas y los precios que encuentra en la salada son:
\begin{itemize}
  \item Jean: 20 pesos
  \item Zapatillas Nike 40 pesos
  \item Calzoncillo de Boca: 5 pesos
  \item Remera de River: 50 pesos
  \item Remera de Independiente: 50 pesos
  \item Remera de Boca: 40 pesos
  \item Ojotas: 15 pesos
  \item Medias con la foto de Carrió: 13 pesos
  \item Medias de fútbol: 15 pesos
\end{itemize}

Entonces, El Pichuleo necesita un programita que le deje ir comprando y siempre le avise cuánta plata le queda. Ejemplo, por pantalla le aparece:

\begin{lstlisting}
Pichu, te quedan 100 pesos. Que te queres comprar?
1) Jean
2) Zapatillas Nike
3) Calzoncillo de Boca
4) Remera de River
5) Remera de Independiente
6) Remera de Boca
7) Ojotas 
8) Medias con la foto de Carrio
9) Medias de futbol
0) Salir
\end{lstlisting}
Si Pichu elige 2, nuestro programa hace lo siguiente:
\begin{lstlisting}
Pichu, te compraste unas zapatillas Nike
Pichu, te quedan 60 pesos. Que te queres comprar?
1) Jean
2) Zapatillas Nike
3) Calzoncillo de Boca
4) Remera de River
5) Remera de Independiente
6) Remera de Boca
7) Ojotas 
8) Medias con la foto de Carrio
9) Medias de futbol
0) Salir
\end{lstlisting}

Si se quiere comprar algo que no le alcanza el programa le dice:
\begin{lstlisting}
Pichu, no te alcanza la plata para eso
\end{lstlisting}
y sigue hasta que se elige 0 que es salir.